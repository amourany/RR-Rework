\chapter*{Introduction}
\addcontentsline{toc}{chapter}{Introduction}

	Nous devons proposer aux clients une implémentation du jeu Rasende Roboter sur Mac et Linux. Le programme doit comprendre une partie en solo comprenant les règles basiques du jeu. Les clients ont souhaité ajouter une fonctionnalité qui permet de résoudre le jeu et la possibilité de jouer en réseau.

	Rasende Roboter est un jeu de société allemand créé par Alex Randolph et illustré par Franz Vohwinkel, édité en 1999 par Hans im Glück / Tilsit. Le jeu est composé d'un plateau, de tuiles représentant chacune une des cases du plateau, et de pions appelés « robots ». La partie est décomposée en tours de jeu, un tour consistant à déplacer les robots sur un plateau afin d'en amener un sur l'une des cases du plateau. Les robots se déplacent en ligne droite et avancent toujours jusqu'au premier mur qu'ils rencontrent. On peut aussi bien y jouer seul qu'à un grand nombre de participants.
	
	Ce projet s'inscrit dans le cadre de l'UE projet de programmation du second semestre de la première année de Master Informatique. Il est réalisé par une équipe de cinq étudiants du Master spécialité Génie Logiciel, trois étudiants (Olivier Braïk, Gaëtan Lussagnet et Dimitri Ranc) sont en option Programmation multi-coeur et GPU, les deux autres (Alexandre Delesse et Alexandre Mourany) sont en option Administration réseaux. Le projet s'étale sur la totalité du semestre.